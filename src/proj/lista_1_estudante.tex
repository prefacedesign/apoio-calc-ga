\documentclass[main_estudante.tex]{subfiles}

\begin{document}

\chapter{Revisão para integrais por partes}

\section{Apresentação}

O objetivo deste capítulo é promover uma revisão de alguns tópicos e habilidades que são necessárias para desenvolver a técnica chamada \textbf{integração por partes} utilizada em MA111.

Você notará uma diferença grande o tom adotado neste capítulo e nos próximos em relação ao tom dos anteriores. Isso se deve ao fato de a abordagem ter intencionalente mudado: ao invés de revisitarmos tópicos do Ensino Médio de um ponto de vista mais conceitual visando oferecer suporte para as disciplinas Cálculo Diferencial e Integral e Geometria Analítica, faremos uma revisão mais focada em tópicos que já foram abordados nessas disciplinas e serão utilizados intensamente em tópicos vindouros.

Além da mudança no conteúdo, a entensão do capítulo também mudou: agora está mais curto. A intenção é que você posse resolver todas as questões aqui presentes em um único encontro da tutoria e utilize o tempo restante para resolver, com auxílio do tutor, exercícios das listas oficiais das disciplinas.

Por fim, retiramos as questões diagnósticas e a auto-avaliação porque os tópicos que serão tratados aqui são novidade para todos os estudantes e esperamos que todos vocês resolvam o capítulo na íntegra.

\section{Quando usar}

Este capítulo deve ser resolvido logo antes do professor de MA111 abordar integração por partes ou logo em seguida.

\section{Conteúdos anteriores}

O conteúdo central desta revisão é a regra do produto, ou seja, não tem problema começar a resolver as questões se você não estiver tão seguro em relação a esse tópico. Porém, os tópicos abaixo são pré-requisitos para este capítulo:
\begin{itemize}
 \item Derivada das funções exponenciais, polinomiais, trigonométricas e logarítmicas.
\end{itemize}

Esses tópicos não serão cobertos durante as atividades de tutoria. Se você acha que não sabe o suficiente sobre algum deles, sugerimos que procure material de apoio antes de começar a resolver as questões desse capítulo.

\newpage

\section{Questões}

A regra do produto foi introduzida como uma técnica para obter a derivada de funções que podem ser vistas como um produto de duas funções mais simples como, por exemplo, $f(x)=x^3.2^x$. Essa função pode ser vista como o produto das funções $g(x)=x^3$ e $h(x)=2^x$, ambas simples de derivar isoladamente.

A regra do produto nos diz que:

\begin{shaded*}
Seja $f(x)=g(x).h(x)$, então $f'(x)=g'(x).h(x)+g(x).h'(x)$.
\end{shaded*}

Aplicando ao nosso exemplo, temos que $g'(x)=3x^2$ e $h'(x)=ln(2).2^x$, portanto:

\begin{align*}
f'(x) &= g'(x).h(x)+g(x).h'(x) \\
 &= (3x^2) . (2^x) + (x^3) . (ln(2).2^x) \\
 &= 3x^2.2^x + ln(2).x^3.2^x
\end{align*}

O objetivo desta lista é relembrar essa técnica e praticar a fluência com ela. Também veremos como utilizá-la para a otenção de integrais que não podem ser calculadas diretamente.

\subsection*{Regra do produto}

Vamos começar praticando a regra do produto.

\begin{questao}
Use a regra do produto para calcular a derivada das seguintes funções.
\begin{enumerate}[a)]
\item $f(x)=x . e^x$
\item $g(x)=x . \cos(x)$
\item $h(x)=x . \sin(x)$
\item $i(x)=x. ln(x)$
\item $j(x)=x^2 . ln(x)$
\end{enumerate}
\end{questao}

\subsection*{Uma integral não elementar}

Na questão anterior bastou aplicar a regra do produto para obtermos a derivada de $f(x)=x . e^x$, que é dada por $f'(x)=e^x+x.e^x$. Será que isso nos ajuda a calcular a integral de $f(x)$, ou seja, $\int x . e^x dx$?

Note primeiro que $\int x . e^x dx$ não pode ser resolvida através de uma troca de variáveis (teríamos que fazer $u=e^x$, o que criaria um logaritmo ao substituirmos $x$). Como claramente se trata de um produto de duas funções que isoladamente são muito simples de derivar e integrar, vamos tentar usar o resultado obtido na item a acima.

Para tanto, note que a segunda parcela de $f'(x)=e^x+x.e^x$ é igual à função que queremos integrar. Portanto, vamos isolar essa parte, obtendo $f'(x)-e^x=x.e^x$ ou, finalmente, $x.e^x=f'(x)-e^x$. Agora, vamos integrar os dois lados da igualdade.

\begin{align*}
\int x.e^x dx &= \int (f'(x)-e^x) dx && \text{Usando propriedades das integrais}\\
\int x.e^x dx &= \int f'(x) dx - \int e^x dx && \text{Sabemos que} \int f'(x) dx = f(x)+c_1 \\
\int x.e^x dx &= f(x)+c_1 - \int e^x dx && \text{Calculando uma das integrais}\\
\int x.e^x dx &= f(x)+c_1 - e^x+c_2 && \text{Lembre-se de que } f(x)=x . e^x\\
\int x.e^x dx &= x . e^x+c_1 - e^x+c_2 && \text{Combinando as constantes} \\
\int x.e^x dx &= x . e^x - e^x+c
\end{align*}

Na última linha acima obtemos a integral desejada. Em resumo, o que fizemos foi isolar a expressão que queríamos integrar no resultado da aplicação da regra do produto e, depois, integramos os dois lados da igualdade.

\begin{questao}
Façamos o mesmo com o resultado do item b da primeira questão para calcular $\int x . \sin(x) dx$.
\begin{enumerate}[a)]
\item Isole $x . \sin(x)$ no resultado obtido no item b da primeira questão.
\item Integre os dois lados da igualdade obtendo $\int x . \sin(x) dx$.
\end{enumerate}
\end{questao}

\begin{questao}
Faça o mesmo com o resultado do item c para calcular $\int x . \cos(x) dx$
\end{questao}

\subsection*{Logaritmos}

Você já notou que apesar de ser uma função simples você ainda não calculou a integral de $a(x)=ln(x)$? 

\begin{questao}
Use o resultado do item d da primeira questão para calcular $\int ln(x) dx$.
\end{questao}

Agora que você já praticou esse processo algumas vezes, tente resolver a questão a seguir.

\begin{questao}
Use o resultado do item e da primeira questão obter a integral de uma função não elementar.
\begin{enumerate}[a)]
\item Qual é essa função?
\item Qual é a sua integral?
\end{enumerate}
\end{questao}

\subsection*{Um novo caso}

Veja que na questão anterior, começamos aplicando a regra do produto à função $j(x)=x^2 . ln(x)$ e terminamos obtendo a integral de $x.ln(x)$. Na questão 4, começamos aplicando a regra do produto à função $i(x)=x . ln(x)$ e terminamos obtendo a integral de $ln(x)$. Você já deve ter notado um padrão nesses resultados, mas se ainda não, aplique a regra do produto para obter a derivada da função $t(x)=x^3. ln(x)$ e observe as potências dos termos obtidos.

\begin{questao}
Obtenha $\int x^5 . ln(x) dx$.
\end{questao}

\subsection*{Duas vezes}

Em alguns casos, duas regras do produto aplicadas a funções diferentes podem ser combinadas gerando resultado úteis, como acontecerá na questão a seguir.

\begin{questao}
Vamos tentar obter $\int e^x.\cos(x) dx$.
\begin{enumerate}[a)]
\item Comece aplicando a regra do produto à função $f(x)=e^x.\cos(x)$ e integrando os dois lados da igualdade.
\item Faça o mesmo processo com a função $g(x)=e^x.\sin(x)$.
\item Note que as duas expressões acima possuem vários termos em comum. Some as duas igualdades e tente isolar os termos de modo a obter uma expressão para $\int e^x.\cos(x) dx$ que não envolva outras integrais.
\end{enumerate}
\end{questao}

\section{Mensagem final}

O processo que você utilizou para resolver todas as questões anteriores é a justificativa por trás da técnica de integração conhecida como \textbf{integral por partes} que será discutida pelo seu professor de MA111. Esse processo pode ser condensado através da fórmula mostrada abaixo.

\begin{shaded*}
$$\int f(x).g'(x)dx = f(x).g(x)-\int f'(x)g(x)dx$$
\end{shaded*}

Essa fórmula facilita o uso dessa técnica no cálculo de integrais, porém, ela nada mais é do que uma síntese do que fizemos anteriormente: regra do produto, isolamento da parcela que interessa e integração dos dois lados da igualdade.

Nossa sugestão é que agora você leia a seção 7.1 do livro \sugestao{Calculus} até o final do quarto exemplo e então resolva os exercícios da lista oficial de MA111 sobre integração por partes.

Aproveite a disponibilidade do tutor e a introdução feita acima para ter certeza de que você compreende bem o uso dessa técnica. Sugerimos que você priorize o entendimento do processo no lugar da quantidade de questões resolvidas e deixe essa parte para um segundo momento quando estiver estudando depois da tutoria.

\newpage

\section{Gabarito}

Confira as respostas para as questões e \textbf{não se esqueça de registrar o seu progresso}.

\noindent\textbf{Questão 1:} a) $f'(x)=e^x+xe^x$, b) $g'(x)=\cos(x)-x\sin(x)$, c) $h'(x)=\sin(x)+x\cos(x)$, d) $i'(x)=ln(x)+1$, e) $i'(x)=2x.ln(x)+x$.

\noindent\textbf{Questão 2:} a) $x\sin(x)=\cos(x)-g'(x)$, b) $\int x\sin(x) dx = \sin(x)-x\cos(x)+c$.

\noindent\textbf{Questão 3:} $\int x\cos(x) dx = \cos(x)+x\sin(x)+c$.

\noindent\textbf{Questão 4:} $\int ln(x) dx = x.ln(x)-x+c$.

\noindent\textbf{Questão 5:} a) $f(x)=x.ln(x)$, b) $\int x.ln(x) dx = \frac{x^2 ln(x)}{2}-\frac{x^2}{4}+c$.

\noindent\textbf{Questão 6:} $\int x^5.ln(x) dx = \frac{x^6 ln(x)}{6}-\frac{x^6}{36}+c$.

\noindent\textbf{Questão 7:} a) $e^x.\sin(x)= \int e^x\sin(x)dx + \int e^x\cos(x)dx$, b) $e^x.\cos(x)= \int e^x\cos(x)dx - \int e^x\sin(x)dx$, c) $\int e^x\cos(x)dx = \frac{e^x\cos(x)+e^x\sin(x)}{2}$.

\section{Registro de progresso}

Essa parte por enquanto fica com conteúdo vazio até que seja decidido como será feito o controle do progresso.

\vspace{5cm}

\end{document}
