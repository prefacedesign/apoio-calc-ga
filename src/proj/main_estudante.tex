%possibilidades: aluno / professor / folha de avaliações

%\newcommand{\versaoApostila}{aluno}
\newcommand{\versaoApostila}{tutor}

% 0 - ambos
% 1 - apenas alunos
% 2 - apenas tutores
% 3 - para folha avaliações
\def\destinacaoConteudo{0}
\def\kParaAmbos{0}
\def\kParaAlunos{1}
\def\kParaTutores{2}
\def\kParaFolhaAvaliacoes{3}


\documentclass[10pt,openany,twoside=semi]{book}

%encoding
%--------------------------------------
\usepackage[utf8]{inputenc}
\usepackage[T1]{fontenc}
\usepackage{amsthm}
\usepackage{xcolor}
\usepackage{gensymb} %simbolo de grau
\usepackage{amssymb}
\usepackage{comment}
\usepackage{etoolbox}
%--------------------------------------

%formatting
\usepackage{xltxtra} %coisas do xelatex incluindo a possibilidade de usar fontes OpenType para o texto.
\usepackage[strict]{changepage} %permite mudar as margens no meio do documento
\usepackage{lipsum}
\usepackage{calc}
\usepackage{tabto}
\usepackage{background} %para fazer a linha vertical
 
%Portuguese-specific commands
%--------------------------------------
\usepackage[portuguese]{babel}
%--------------------------------------

\usepackage[shortlabels]{enumitem}
\usepackage{graphicx}
\usepackage{wrapfig}
\usepackage{amsmath}
\usepackage{subfiles} %suporte a multiplos arquivos
\graphicspath{{img/}} %pasta das imagens
\usepackage{fancyhdr} %para formatar o rodape

\generalcomment{destinacao}{}{}

\newcommand{\paraTutores}{%
	\ifnum\destinacaoConteudo=\kParaTutores
		\def\eraPraTutores{1}
	\else
		\def\eraPraTutores{0}
	\fi
	
	\def\destinacaoConteudo{\kParaTutores}	

	\ifnum\strcmp{\versaoApostila}{tutor}=0
		\color{black}
		\ArchivoMediumNine
		\leftskip=-1.80\gridM
	\fi
	
	\ifnum\strcmp{\versaoApostila}{aluno}=0
		\begin{comment}
	\fi
}

\newcommand{\paraAlunos}{%
	\ifnum\strcmp{\versaoApostila}{tutor}=0
		\color{gray}
		\setmainfont{Spectral}
		\leftskip=0pt
	\fi
	\def\destinacaoConteudo{\kParaAlunos}
}

\newcommand{\paraAmbos}{%
	\def\destinacaoConteudo{\kParaAmbos}
	\ifnum\strcmp{\versaoApostila}{tutor}=0
		\color{black}
		\setmainfont{Spectral}
		\leftskip=0pt
	\fi
}

\newdimen\gridM\gridM=24.70pt
\newdimen\gridL\gridL=13.228pt
% um módulo = 24.70pt
% uma linha = 13.228pt





\setmainfont{Spectral}
\linespread{1.10} %vira uma line height de 13.228pt
\usepackage{parskip}
\setlength{\parindent}{0pt}
\setlength{\parskip}{1\gridL}
\definecolor{titlecolor}{cmyk}{0,0.44,1,0}
\definecolor{questaocolor}{cmyk}{0,0.44,1,0}
\definecolor{boxbg}{cmyk}{0,0.04,0.1,0}

\setsansfont{Archivo}[
  FontFace={archivor}{n}{Font=* Regular},
  FontFace={archivom}{n}{Font=* Medium},
  FontFace={archivosb}{n}{Font=* SemiBold},
  FontFace={archivob}{n}{Font=* Bold},
]

\newfontfamily\ArchivoBlackFourteen[SizeFeatures={Size=14}]{Archivo Black}
\newfontfamily\ArchivoBlackTen[SizeFeatures={Size=10}]{Archivo Black}
\newfontfamily\ArchivoMediumTitle[SizeFeatures={Size=36}]{Archivo}
\newfontfamily\ArchivoMediumNine[SizeFeatures={Size=9}]{Archivo}
\newfontfamily\ArchivoMediumEight[SizeFeatures={Size=7.8}]{Archivo}
\newfontfamily\BioRhymeExpandedChapter[SizeFeatures={Size=170}]{BioRhyme Expanded}
\newfontfamily\BioRhymeExpandedQuestao[SizeFeatures={Size=10}]{BioRhyme Expanded}
\usepackage[euler-digits,euler-hat-accent]{eulervm}
\usepackage[normalem]{ulem}

\TabPositions
	{%
	 0.5\gridM ,
	 1.0\gridM ,
	 1.5\gridM ,
	 2.0\gridM ,
	 2.5\gridM ,
	 3.0\gridM ,
	 3.5\gridM ,
	 4.0\gridM ,
	 4.5\gridM ,
	 5.0\gridM ,
 	 5.5\gridM ,
	 6.0\gridM ,
 	 6.5\gridM ,
	 7.0\gridM ,
 	 7.5\gridM ,
	 8.0\gridM ,
	 8.5\gridM ,
	 9.0\gridM ,
 	 9.5\gridM ,
	 10.0\gridM ,
 	 10.5\gridM ,
	 11.0\gridM ,
 	 11.5\gridM ,
 	 12.0\gridM ,
 	 12.5\gridM ,
 	 13.0\gridM ,
 	 13.5\gridM ,
 	 14.0\gridM ,
 	 14.5\gridM ,
 	 15.0\gridM ,
 	 15.5\gridM ,
 	 16.0\gridM ,
 	 16.5\gridM
	 }

%novos ambientes que eu criei
%--------------------------------------
\newtheorem*{teorema}{Teorema}
\newtheoremstyle{estiloQuestao}
	{\gridL}
	{0pt}
	{\setlist[enumerate]{font=\bfseries, leftmargin=0pt, labelsep=1.0\gridM}}
	{-2\gridM}
	{\BioRhymeExpandedQuestao\color{questaocolor}}
	{\hspace{-5pt}}
	{}
	{\hbox to 2\gridM{\thmname{#1}\thmnumber{#2}} \thmnote{#3}}
\theoremstyle{estiloQuestao}
\newtheorem{questao}{Q}[]
\theoremstyle{plain}
\newtheorem{diagnostico}{Questão}
\newtheorem*{reflita}{Reflita}
\newtheorem*{resolvida}{Questão resolvida}
\newtheorem*{resolva}{Questão do livro-texto}

\DeclareTextFontCommand\sugestao{\textsc}
%--------------------------------------

%elementos visuais
%--------------------------------------







\usepackage{framed} %para o box
\colorlet{shadecolor}{boxbg} %para o box
\usepackage{tcolorbox} %melhor para o box do que o framed
\tcbuselibrary{breakable}


\usepackage{marginnote}
\usepackage{geometry} %margens

\ifnum\strcmp{\versaoApostila}{aluno}=0
 
 \geometry{
 a5paper,
 inner=3.5\gridM,
 outer=3.5\gridM,
 top=3\gridL,
 bottom=6\gridL,
 }
 
 \newcommand{\notaTutor}[1]{ }

\else
 
 \geometry{
 a4paper,
 left=4\gridM,
 right=10.117\gridM,
 top=3\gridL,
 bottom=10\gridL,
 marginparsep=2.5\gridM,
 marginparwidth=7\gridM,
 asymmetric
 }
 
 \normalmarginpar
 
 \newcommand{\notaTutor}[1]{
 	\hspace{0pt}\marginpar{
 		\scriptsize
 		\linespread{1.654} 
 		\ArchivoMediumEight
 		#1}}
 
\fi

\usepackage{titlesec} %titulos para as questoes




%Fazendo as caixinhas de exemplo
\newlength{\identacao}
\setlength{\identacao}{2\gridM-5pt}%

\newenvironment{caixaExemplo}{\begin{tcolorbox}[%
	grow to left by=2\gridM,
	left=2pt,
	grow to right by=2\gridM,
	right=2pt,
	boxrule=0pt,
	arc=0pt,
	colframe=boxbg,
	colback=boxbg,
	breakable
]
\itshape
\leftskip=\identacao
\rightskip=\identacao}{\end{tcolorbox}}


\newcommand{\tituloSubsection}[1]{%
   \ifnum\destinacaoConteudo=\kParaTutores{
   	\leftskip=-1.9\gridM \parskip=0
   	\ArchivoBlackTen #1
   	
}
   \else { \ArchivoBlackTen #1 }
   \fi
   
}

\titleformat{name=\subsection}{}{}{0em}{\tituloSubsection} %2 módulos ====>>> Muito estranho se desloca prá baixo ele ferra com a divisória

\def\divisoria{\bgroup \markoverwith{\lower3.5\p@\hbox{\sixly \textcolor{titlecolor}{\char58}}}\ULon}
\font\sixly=lasyb10
\def\divisoriapadrao{\divisoria{\hspace{14\gridM}}}

%\titleformat{name=\section}{\ArchivoBlackFourteen}{}{0em}{
%{\hspace{-1pt}\divisoriapadrao \vspace{10pt}}\linebreak
%\raggedright
%\leftskip=-\identacao
%\rightskip=-\identacao
%\parindent=0pt
%\hangindent=0pt
%\color{titlecolor}\MakeUppercase
%} %2 módulos


{

\baselineskip15pt
\lineskiplimit-20pt

}

\newcommand{\tituloComOndulado}[1]{%
\pagebreak[3]
\ifnum\destinacaoConteudo=\kParaTutores
{%
   \leftskip=-1.80\gridM 
   \ArchivoBlackTen \MakeUppercase{#1}
   
}
\else
  \begin{tcolorbox}[%
	grow to left by=2\gridM,
	left=0pt,
	grow to right by=2\gridM,
	right=0pt,
	boxrule=0pt,
	arc=0pt,
	colframe=boxbg,
	colback=white
  ]
	\ArchivoBlackFourteen
	{\hspace{-1pt}\divisoriapadrao \vspace{10pt}}\linebreak

	\vspace{-10pt}
	\parindent=0pt
	\hangindent=0pt
	\raggedright
	\textcolor{titlecolor}{\MakeUppercase{#1}}
  \end{tcolorbox}
\fi
\nopagebreak[4]
}

%aqui

\newcommand{\caixatitulao}[1]{%
\vspace{-9\gridL}
\begin{tcolorbox}[%
	grow to left by=2\gridM,
	left=0pt,
	grow to right by=2\gridM,
	right=0pt,
	boxrule=0pt,
	arc=0pt,
	colframe=boxbg,
	colback=white,
	text height=12\gridL,
	breakable
]

{\color{titlecolor}\BioRhymeExpandedChapter\hspace{-0.5\gridM}\thechapter}
	\vspace{-6\gridL}
	\begin{flushleft}
		\par\leftskip=2\gridM \ArchivoMediumTitle #1
	\end{flushleft}
\end{tcolorbox}
}

\titleformat{name=\chapter}[display]{}{}{0em}{\caixatitulao}[\vspace{-6\gridL}]
\titleformat{name=\section}[display]{}{}{0em}{\tituloComOndulado}[\vspace{-3\gridL}]

\setlist[enumerate]{leftmargin=0pt, labelsep=0.5\gridM}
\setlist[itemize]{leftmargin=0pt, labelsep=0.5\gridM}
%texto alinhado a esquerda
\raggedright
%evita de ele preencher os espaços verticais
\raggedbottom

%\newcommand{\hsp}{\hspace{20pt}}
%\usepackage{blindtext}
%\titleformat{\chapter}[hang]{\Huge\bfseries}{\thechapter\hsp\textcolor{titlecolor}{|}\hsp}{0pt}{\Huge\bfseries}

%


%\titleformat{\subsection}[display]{\bfseries}{}{}
%{ % before-code
%    \rule{\textwidth}{1pt}
%    \vspace{1ex}
%    \centering
%} 
%[ % after-code
%\vspace{-1ex}%
%\rule{\textwidth}{0.2pt}
%] 
%--------------------------------------


\ifnum\strcmp{\versaoApostila}{aluno}=0
	\newcommand{\titulo}{Material do estudante - Tutoria MA111 e MA141}
\else
	\newcommand{\titulo}{Material do tutor - Tutoria MA111 e MA141}
\fi

\title{\titulo}
\author{Leonardo Barichello}
\date{\today}



\begin{document}


\backgroundsetup{contents={}}
\maketitle
%Rodape - PARA FUNCIONAR CORRETAMENTE TEM QUE SER COLOCADO DEPOIS DO MAKETITLE
%---------------------------------------
\fancypagestyle{plain}{%
  \fancyhf{}%
  \renewcommand{\headrulewidth}{0pt}
  \renewcommand{\footrulewidth}{0pt}
  \fancyfoot[LE,LO]{%
	\ifnum\strcmp{\versaoApostila}{aluno}=0
		\vspace{0.5\gridL}
	\else
		\vspace{2.5\gridL}
	\fi
	\hbox to 16\gridM{\hspace{-2\gridM}\hbox to 14\gridM{\rule[0.5\gridL]{14\gridM + 2.5pt}{0.5pt}}}
	\hbox to 14\gridM{\hspace{-2\gridM}\hbox to 2\gridM{\thepage\hfill}\hfill}
  }
}


\pagestyle{fancy}
\fancyhf{}

\fancyfoot[LO]{%
	\ifnum\strcmp{\versaoApostila}{aluno}=0
		\vspace{0.5\gridL}
	\else
		\vspace{2.5\gridL}
	\fi
	\hbox to 16\gridM{\hspace{-2\gridM}\hbox to 14\gridM{\rule[0.5\gridL]{14\gridM + 2.5pt}{0.5pt}}}
	\hbox to 14\gridM{\hspace{-2\gridM}\hbox to 2\gridM{\thepage\hfill}\textsc{\MakeLowercase{\leftmark}}\hfill}
}
\fancyfoot[LE]{%
	\ifnum\strcmp{\versaoApostila}{aluno}=0
		\vspace{0.5\gridL}
	\else
		\vspace{2.5\gridL}
	\fi
	\hbox to 16\gridM{\hspace{-2\gridM}\hbox to 14\gridM{\rule[0.5\gridL]{14\gridM +2.5pt}{0.5pt}}}
	\hbox to 14\gridM{\hspace{-2\gridM}\hbox to 2\gridM{\thepage\hfill}\textsc{\MakeLowercase{\titulo}}\hfill}
}

\renewcommand{\headrulewidth}{0pt}
\renewcommand{\footrulewidth}{0pt}
\renewcommand{\chaptermark}[1]{%
\markboth{#1}{}}

%faz a linha vertical
\ifnum\strcmp{\versaoApostila}{tutor}=0
 \SetBgScale{1}
 \SetBgAngle{0}
 \SetBgColor{lightgray}
 \SetBgContents{\rule{.5pt}{\paperheight}} 
 \SetBgHshift{16.117\gridM -\paperwidth / 2 }
\fi

%---------------------------------------

\chapter{Introdução}

Este material foi desenvolvido especificamente para os estudantes que foram aprovados em algum curso de Exatas tendo obtido nota em Matemática no vestibular ou no ENEM abaixo do ideal.

A experiência mostra que esses estudantes provavelmente terão dificuldades nas duas disciplinas de Matemática que são obrigatórias em praticamente todos os cursos de Exatas: Cálculo Diferencial e Integral I e Geometria Analítica. Com isso em mente, a Unicamp criou a iniciativa Tutoria em Matemática. Essa iniciativa visa oferecer suporte especificamente concebido para esses estudantes de modo a facilitar a transição entre o Ensino Médio e o Ensino Superior.

Temos três grandes objetivos com esse projeto. Primeiro, reforçar alguns tópicos matemáticos importantes do Ensino Médio. Segundo, reforçar a conexão entre o que você aprendeu nos últimos anos com o que será ensinado nessas disciplinas. Terceiro, criar um espaço para que o estudante estude ativamente com suporte de um tutor.

\tituloComOndulado{Isto é um teste}

%\end{comment}
\paraAlunos

A tutoria consiste em 3 horas de atividades semanais (cheque no seu horário como elas estão distribuídas), sempre em pequenos grupos acompanhados por um tutor. Todos os encontros seguirão atividades propostas em cadernos especialmente desenvolvidos de acordo com o objetivo da iniciativa e acompanhando de perto o conteúdo das disciplinas principais.


%\end{comment}
\paraTutores

\section{TESTE texto apenas para tutores}

A tutoria consiste em 3 horas de atividades semanais (cheque no seu horário como elas estão distribuídas), sempre em pequenos grupos acompanhados por um tutor. Todos os encontros seguirão atividades propostas em cadernos especialmente desenvolvidos de acordo com o objetivo da iniciativa e acompanhando de perto o conteúdo das disciplinas principais.

A tutoria consiste em 3 horas de atividades semanais (cheque no seu horário como elas estão distribuídas), sempre em pequenos grupos acompanhados por um tutor. Todos os encontros seguirão atividades propostas em cadernos especialmente desenvolvidos de acordo com o objetivo da iniciativa e acompanhando de perto o conteúdo das disciplinas principais.

 \subsection{Esta é uma subseção}

A tutoria consiste em 3 horas de atividades semanais (cheque no seu horário como elas estão distribuídas), sempre em pequenos grupos acompanhados por um tutor. Todos os encontros seguirão atividades propostas em cadernos especialmente desenvolvidos de acordo com o objetivo da iniciativa e acompanhando de perto o conteúdo das disciplinas principais.

A tutoria consiste em 3 horas de atividades semanais (cheque no seu horário como elas estão distribuídas), sempre em pequenos grupos acompanhados por um tutor. Todos os encontros seguirão atividades propostas em cadernos especialmente desenvolvidos de acordo com o objetivo da iniciativa e acompanhando de perto o conteúdo das disciplinas principais.

%\end{comment}
\paraAlunos

\section{TESTES}

\lipsum[1]




\begin{caixaExemplo}

\lipsum[1]

$$\begin{pmatrix} 2 && 1 \\ -3 && 5 \end{pmatrix} + \begin{pmatrix} 4 && 10 \\ 5 && -1 \end{pmatrix} + \begin{pmatrix} 4 && 10 \\ 5 && -1 \end{pmatrix} + \begin{pmatrix} 4 && 10 \\ 5 && -1 \end{pmatrix} = \begin{pmatrix} 6 && 11 \\ 2 && 4  \end{pmatrix}$$


\end{caixaExemplo}

\begin{caixaExemplo}
Exemplos

$$\begin{pmatrix} 2 && 1 \\ -3 && 5 \end{pmatrix} + \begin{pmatrix} 4 && 10 \\ 5 && -1 \end{pmatrix} + \begin{pmatrix} 4 && 10 \\ 5 && -1 \end{pmatrix} + \begin{pmatrix} 4 && 10 \\ 5 && -1 \end{pmatrix} = \begin{pmatrix} 6 && 11 \\ 2 && 4  \end{pmatrix}$$


\end{caixaExemplo}

%Mudar as margens da página\begin{adjustwidth}{-20pt}{-20pt}

\textbf{Vá com calma}. A quantidade de atividades propostas foi pensada para que haja tempo para resolver todas as questões durante os encontros. Portanto, não corra. Leia atentamente as questões e os textos explicativos antes e depois delas. Uma grande parte da aprendizagem esperada vem dessas explicações. Caso você não termine algum capítulo, não se preocupe. É mais importante que você compreenda cada tópico visto do que chegue ao final apressadamente.


%\end{comment}
\paraAmbos
 
\notaTutor{Mas, ao invés de respostas prontas, procure sugestões ou esclarecimentos que lhe permitam resolver as questões e entender os conceitos de maneira independente.} \textbf{Seja ativo}. Ao ler o material, tenha certeza de que você entendeu o conteúdo. Volte e cheque as referências, refaça cálculos se for necessário e faça anotações. Jamais deixe de registrar o processo de resolução de uma questão de modo que você consiga relê-lo no futuro se desejar. Recomendamos que você faça anotações tanto no texto quanto nas suas resoluções que lhe permitam tanto entender  melhor o que foi feito quanto re-entender caso um dia você o consulte novamente.


\textbf{Pergunte}. Peça ajuda aos colegas e ao seu tutor caso não tenha conseguido entender alguma coisa. Não se sinta inibido, pois outros colegas podem ter as mesmas dúvidas que você ou serem capazes de ter explicar algo a partir de um ponto muito parecido com o que você está agora. Mas, ao invés de respostas prontas, procure sugestões ou esclarecimentos que lhe permitam resolver as questões e entender os conceitos de maneira independente.

\notaTutor{Mas, ao invés de respostas prontas, 
$$ 1 = x^2 $$
procure sugestões ou esclarecimentos que lhe permitam resolver as questões e entender os conceitos de maneira independente.}\textbf{Mantenha o engajamento.} Se você não conseguir terminar algum capítulo, não se preocupe. Este material foi concebido pensando nessa possibilidade: todo trabalho feito aqui deve te ajudar nas disciplinas principais cedo ou tarde. Por outro lado, se em algum momento o conteúdo parecer inútil ou muito fácil, mantenha o engajamento pois os tópicos foram cuidadosamente escolhidos e vocẽ notará o efeito do material em breve.

\textbf{Registre o seu progresso}. Não deixe de registrar o seu progresso ao final de cada capítulo. Isso é importante para podermos avaliar o quão bem a iniciativa está funcionando e para que você não deixe de completar as atividades propostas.

\textbf{Reflita}. Haverão questões explicitamente focadas em lhe fazer refletir sobre os tópicos discutidos e oportunidades para que você pense sobre o que você sabe, o quanto aprendeu e o que pode fazer para melhorar. Essas habilidades são importantes, não as menospreze! 

1 \tab 2 \tab 3 \tab 4 \tab 5 \tab 6 \tab 7 \tab 8 \tab 9 \tab 0 \tab 1 \tab 2 \tab 3 \tab 4 \tab 5 \tab 6 \tab 7 \tab 8 \tab 9 \tab 0 \tab 1 \tab 2 \tab 3 \tab 4 \tab 5 \tab 6 \tab 7 \tab 8 \tab 9 \tab 0 \tab 1 \tab 2 \tab 3 \tab 4 \tab 5 \tab 6 \tab 7 \tab 8 \tab 9 \tab 0 \tab 1 \tab 2 \tab 3 \tab 4 \tab 5 \tab 6 \tab 7 \tab 8 \tab 9 \tab 0 \tab 1 \tab 2 \tab 3 \tab 4 \tab 5 \tab 6 \tab 7 \tab 8 \tab 9 \tab 0 \tab 


\textbf{Não negligencie as disciplinas principais}. O objetivo da tutoria é te ajudar com as duas disciplinas principais e não ser mais um disciplinas por si só. Use o tempo da tutoria para a tutoria, mas não retire tempo de estudo das principais para investir na tutoria. Se a carga de trabalho estiver demais, conversa com seu tutor.

%\paraAmbos

\section{O que esperamos de você}

\textbf{Vá com calma}. A quantidade de atividades propostas foi pensada para que haja tempo para resolver todas as questões durante os encontros. Portanto, não corra. Leia atentamente as questões e os textos explicativos antes e depois delas. Uma grande parte da aprendizagem esperada vem dessas explicações. Caso você não termine algum capítulo, não se preocupe. É mais importante que você compreenda cada tópico visto do que chegue ao final apressadamente.
 
\notaTutor{Mas, ao invés de respostas prontas, procure sugestões ou esclarecimentos que lhe permitam resolver as questões e entender os conceitos de maneira independente.} \textbf{Seja ativo}. Ao ler o material, tenha certeza de que você entendeu o conteúdo. Volte e cheque as referências, refaça cálculos se for necessário e faça anotações. Jamais deixe de registrar o processo de resolução de uma questão de modo que você consiga relê-lo no futuro se desejar. Recomendamos que você faça anotações tanto no texto quanto nas suas resoluções que lhe permitam tanto entender  melhor o que foi feito quanto re-entender caso um dia você o consulte novamente. 

\textbf{Pergunte}. Peça ajuda aos colegas e ao seu tutor caso não tenha conseguido entender alguma coisa. Não se sinta inibido, pois outros colegas podem ter as mesmas dúvidas que você ou serem capazes de ter explicar algo a partir de um ponto muito parecido com o que você está agora. Mas, ao invés de respostas prontas, procure sugestões ou esclarecimentos que lhe permitam resolver as questões e entender os conceitos de maneira independente.

\textbf{Mantenha o engajamento.} Se você não conseguir terminar algum capítulo, não se preocupe. Este material foi concebido pensando nessa possibilidade: todo trabalho feito aqui deve te ajudar nas disciplinas principais cedo ou tarde. Por outro lado, se em algum momento o conteúdo parecer inútil ou muito fácil, mantenha o engajamento pois os tópicos foram cuidadosamente escolhidos e vocẽ notará o efeito do material em breve.

\textbf{Registre o seu progresso}. Não deixe de registrar o seu progresso ao final de cada capítulo. Isso é importante para podermos avaliar o quão bem a iniciativa está funcionando e para que você não deixe de completar as atividades propostas.

\textbf{Reflita}. Haverão questões explicitamente focadas em lhe fazer refletir sobre os tópicos discutidos e oportunidades para que você pense sobre o que você sabe, o quanto aprendeu e o que pode fazer para melhorar. Essas habilidades são importantes, não as menospreze! 

\textbf{Não negligencie as disciplinas principais}. O objetivo da tutoria é te ajudar com as duas disciplinas principais e não ser mais um disciplinas por si só. Use o tempo da tutoria para a tutoria, mas não retire tempo de estudo das principais para investir na tutoria. Se a carga de trabalho estiver demais, conversa com seu tutor.

\section{Estrutura do material}

A maioria dos capítulos deste material, a partir do próximo, segue a mesma estrutura:

\begin{enumerate}
 \item Apresentação: explicando porque o tópico em questão foi escolhido para o material e em que ele deve te ajudar nas disciplinas de Cálculo Diferencial e Integral e Geometria Analítica;
 \item Pré-requisitos e Auto-avaliação inicial: explicando quais são os pré-requisitos do capítulo, que eventualmente precisam ser estudados antes dos encontros, e oferecendo uma oportunidade para você refletir sobre o seu conhecimento em Matemática;
 \item Avaliação Diagnóstica: para que você inicie as atividades do capítulo em um ponto compatível com o seu conhecimento. Dê o seu melhor e a resolva sozinho. Essa avaliação deve ser resolvida ao final do último encontro do capítulo anterior. Assim, você pode aproveitar os dias antes do próximo encontro para estudar os pré-requisitos e os tópicos que foram difíceis pra você na Avaliação Diagnóstica;
 \item Questões: onde se concentra a maior parte do conteúdo, formado por questões e por texto discutindo os tópicos em pauta;
 \item Rumo ao livro texto: com o objetivo de propor questões ou leituras que explicitamente conectem o trabalho que você acabou de fazer com o livros-texto das disciplinas oficiais;
 \item Gabarito: contém as respostas para quase todas as questões. Use para checar as suas respostas quando você terminar de resolver uma questão, não para copiar a resposta final ou para ``forçar'' o caminho da resolução;
 \item Registro de progresso: para que você registre quais questões resolveu (não importa se certo ou errado). Essa seção é importante para que possamos acompanhar a implementação do projeto e aprimorar o material;
 \item Auto-avaliação final: oferecendo uma oportunidade para você comparar a sua evolução e traçar metas de estudo.
\end{enumerate}

\section{Referências essenciais}

Este material é bastante auto-contido, mas alguns outros livros serão referenciados tanto para sugerir leituras que expliquem tópicos não cobertos pelo material quanto para indicar leituras de aprofundamento ou continuidade.

A lista a seguir contém todas as referências que serão usadas ao longo do material. Sugerimos que você tenha esses materiais disponíveis durante as atividades da tutoria. Todos podem ser encontrados na biblioteca ou na internet.

\begin{itemize}
 \item O livro digital \sugestao{Matemática Básica volume 1}, de Francisco Magalhães Gomes, professor do IMECC. Disponível em http://www.ime.unicamp.br/~chico
 \item O livro digital \sugestao{Matrizes, Vetores e Geometria Analítica}, de Reginaldo J. Santos. Disponível em www.mat.ufmg.br/~regi
 \item O livro físico \sugestao{Álgebra Linear}, de José Luiz Boldrini e outros. Amplamente disponível na biblioteca.
\end{itemize}

Existem também diversos portais com vídeos abordando tópicos de matemática na internet. Enquanto vários deles são bons, alguns não são. A sugestão que fazemos é o Portal do Saber (portaldosaber.obmep.org.br). Ele se destaca pela organização, qualidade dos vídeos, recursos disponíveis e uniformidade do material.

\section{Algumas palavras sobre o Ensino Superior}

Aqui virá o texto da Adriane sobre a transição para o Ensino Superior.

\newpage

\subfile{cap1_estudante.tex} %Matrizes 2x2

\subfile{cap2_estudante.tex} %Potenciais, exp e log

%\subfile{cap3_estudante.tex} %Polinomios

%\subfile{cap4_estudante.tex} %trigonometria e vetores

%\subfile{cap5_estudante.tex} %troca de variaveis

%\subfile{cap6_estudante.tex} %reta e circunferencia

%\subfile{cap7_estudante.tex} %graficos

%\subfile{cap8_estudante.tex} %desigualdades
 
%mudanca de formato no material

%\subfile{lista_1_estudante.tex} %revisao para integrais por partes

%\subfile{lista_2_estudante.tex} %identidades trigonometricas

%\subfile{lista_3_estudante.tex} %fracoes algebricas para fracoes parciais

%teste
 
\end{document}