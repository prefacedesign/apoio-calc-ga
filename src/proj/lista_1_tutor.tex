\documentclass[main_estudante.tex]{subfiles}

\begin{document}

\chapter{Revisão para integrais por partes}

\section{Nota explicativa}

Este capítulo é o primeiro dos três capítulos finais da tutoria. Estes capítulos tem uma proposta diferente dos capítulos anteriores. O motivo dessa mudança está no fato de o conteúdo de ambas as disciplinas-alvo deste material, Cálculo Diferencial e Integral e Geometria Analítica, terem atingido um ponto em que os pré-requisitos são tópicos vistos anteriormente nas próximas disciplinas e não mais tópicos que poderiam ter sido estudados anteriormente no Ensino Médio.

Na verdade, os três capítulos estão focados em tópicos de Cálculo Diferencial e Integral e esse é um outro motivo da mudança, os tópicos em questão são bastante específicos (técnicas de integração) e parece-nos mais razoável propor atividades mais focadas no uso específico que será feito de conteúdos como frações algébricas, identidades trigonométricas e regra do produto. Dessa forma, o ritmo mais dialogado e conceitual adotado anteriormente dá espaço a uma ênfase em aspectos específicos.

Também houveram mudanças na estrutura do material. São elas:

\begin{itemize}
 \item \textbf{Não há mais questões diagnósticas}, já que esperamos que os tópicos aqui tratados serão novos para todos os estudantes. Consequentemente, não há mais um quadro de orientação. Para o estudante não há mai quadro de auto-avaliação;
 \item \textbf{A seção Rumo ao Livro-Texto foi removida}. Ao final das questões, sugerimos a leitura de uma seção e alguns exemplos do livro-texto. A intenção é que após essa leitura, os estudantes estejam prontos para resolver questões das listas de exercícios oficiais da disciplina. Pelo mesmo motivo não sugerimos questões adicionais para o tutor;
 \item \textbf{O capítulo é mais curto} e o tempo restante deve ser usado para auxiliar os estudantes na resolução de questões da lista de exercícios oficial da disciplina.
\end{itemize}

Um detalhe importante continua igual: dê tempo para que os estudantes resolvam as questões com atenção, leiam os textos e compreendam o que estão fazendo. Apesar das questões mais focadas, a tutoria deve continuar focando em entendimento dos conceitos e não para prática de exercícios.

\section{Comentários iniciais}

Este capítulo é focado no uso da regra do produto com o objetivo de calcular integrais usando a técnica chamada de integração por partes. Ao longo do texto, tentaremos focar mais na regra do produto em si e não nas integrais, pois isso será feito pelo professor de MA111.

Nosso objetivo aqui não é sistematizar o método através de fórmulas como $\int udv=uv-\int vdu$. Ao contrário, o que queremos é revisar a regra do produto e deixar claro como ela pode ser usada para calcular algumas integrais que envolvem produtos de funções simples.

Será comum ao longo do capítulo a manipulação algébrica de igualdades envolvendo integrais e derivadas. Seus estudantes provavalmente estranharam esse processo, mas isso ocorre simplemesnte pelo fato de as igualdades envolverem integrais e derivadas. As regras para manipulação de igualdades continuam valendo: somar dos dois lados, mutliplciar dos dois lados a assim por diante. No final das contas, isso é tudo que precisaremos para este cpítulo.

\section{Quando usar}

Certifique-se de que o professor dos estudantes da sua turma esteja prestes a iniciar o tópico \textbf{integração por partes}. Esse é o momento ideal para usar este capítulo.

\section{Questões comentadas}

\begin{questao}
Use a regra do produto para calcular a derivada das seguintes funções.
\begin{enumerate}[a)]
\item $f(x)=x . e^x$
\item $g(x)=x . \cos(x)$
\item $h(x)=x . \sin(x)$
\item $i(x)=x. ln(x)$
\item $j(x)=x^2 . ln(x)$
\end{enumerate}
\end{questao}

O capítulo começa com uma revisão sobre a regra do produto, focada em casos que são normalmente usados como introdução para integração por partes.

\begin{questao}
Façamos o mesmo com o resultado do item b da primeira questão para calcular $\int x . \sin(x) dx$.
\begin{enumerate}[a)]
\item Isole $x . \sin(x)$ no resultado obtido no item b da primeira questão.
\item Integre os dois lados da igualdade obtendo $\int x . \sin(x) dx$.
\end{enumerate}
\end{questao}

\begin{questao}
Faça o mesmo com o resultado do item c para calcular $\int x . \cos(x) dx$
\end{questao}

Dois casos bastante simples de integração por partes. Insista que os estudantes leiam e entendam o processo feito com o item a logo antes desta questão.

\begin{questao}
Use o resultado do item d da primeira questão para calcular $\int ln(x) dx$.
\end{questao}

Uma questão ainda simples, mas sem itens orientando a resolução. Não esperamos que hajam dificuldades específicas com o item.

\begin{questao}
Use o resultado do item e da primeira questão obter a integral de uma função não elementar.
\begin{enumerate}[a)]
\item Qual é essa função?
\item Qual é a sua integral?
\end{enumerate}
\end{questao}

Dessa vez, o estudante não sabe qual integral irá obter ao final do processo. Entretanto, pela sua similaridade com itens anteriores, isso não deve representar um obstáculo para eles.

\begin{questao}
Obtenha $\int x^5 . ln(x) dx$.
\end{questao}

Uma vez um pouco familiarizdos com o processo, essa questão demanda um pouco de reconhecimento de padrões. O texto logo antes da questão sugere um exemplo mais simples que pode ser resolvido antes caso o estudante não saiba o que fazer. Insista na sugestão caso isso ocorra.

\begin{questao}
Vamos tentar obter $\int e^x.\cos(x) dx$.
\begin{enumerate}[a)]
\item Comece aplicando a regra do produto à função $f(x)=e^x.\cos(x)$ e integrando os dois lados da igualdade.
\item Faça o mesmo processo com a função $g(x)=e^x.\sin(x)$.
\item Note que as duas expressões acima possuem vários termos em comum. Some as duas igualdades e tente isolar os termos de modo a obter uma expressão para $\int e^x.\cos(x) dx$ que não envolva outras integrais.
\end{enumerate}
\end{questao}

Essa última questão é um pouco mais sofisticada por combinar resultados de duas regras do produto siferentes. Esse tipo de situação é comum mesmo quando a fórmula de integração por partes é utilizada, portanto. Além disso, o caso ainda nã é tão complicado a ponto de ofuscar o processo por trás da resolução.

\section{Comentários finais}

A abordagem que usamos ao longo deste capítulo não é a mais prática para uso da técnica de integração por partes. Fórmulas como a que foi apresentada no início do capítulo facilitam a aplicação do método e diminuem a necessidade de ``sacadas'', como a que foi necessária na questão 7. Entretanto, isso será feito pelo professor de MA111.

Quando completarem a lista, sugerimos aos estudantes a leitura da seção e de alguns exemplos referente a essa técnica no livro \sugestao{Calculus}. Em seguida, eles devem resolver questões da lista de exercícios da disciplina. Quando chegarem a este ponto, não exija que os estudantes usem o método que usamos ao longo do capítulo, mas sim o método que o professor da disciplina estiver utilizando.

\section{Gabarito}

\noindent\textbf{Questão 1:} a) $3$, b) $3\sqrt{3}/2$, c) $9\sqrt{3}/4$.

\noindent\textbf{Questão 2:} a) $0,98$, b) $0,26$, c) $0,3$, d)$0,9$, e) $0,07$.

\noindent\textbf{Questão 3:} a) $3,76$, b) $9,24$.

\noindent\textbf{Questão 4:} a) $0,78$, b) $1,16$, c) $37\degree$, d) $30\degree$.

\noindent\textbf{Questão 5:} a) $\sqrt{13}$, b) $2\sqrt{13}/13$ e $3\sqrt{13}/13$, c) $0,59$ e $34\degree$.

\noindent\textbf{Questão 6:} a) $\Arrowvert V \Arrowvert = \sqrt{5}$ e $\Arrowvert U \Arrowvert = \sqrt{10}$, b) $63\degree$ e $18\degree$, c) $45\degree$.

\noindent\textbf{Questão 7:} a) $(2,30;1,93)$, b) $(\sqrt{3};1,00)$.

\noindent\textbf{Questão 8:} c) $\sqrt{10}/2$, d) $\frac{5}{2}$.

\noindent\textbf{Questão 9:} a) $2,42$.

\noindent\textbf{Questão 10:} a) $2.301$, b) $1.301$, c) $0.301$.

\noindent\textbf{Questão 11:} $1,93$.

\noindent\textbf{Questão 12:} b) $f^{-1}(x)=\frac{x+8}{4}$, c) $g^{-1}(x)=\frac{3}{x-2}+1$, d) $h^{-1}(x)=\frac{10^{x-3}}{2}$, e) $p^{-1}(x)=\frac{\arcsin(y)+10}{3}$, f) $c^{-1}(x)=log_2 (y)-5$.

\end{document}
