\documentclass[main_estudante.tex]{subfiles}

\begin{document}

\chapter{Revisão para frações parciais}

\section{Apresentação}

A motivação para essa técnica de integração vem do fato de que integrais do tipo $\int \frac{c}{ax+b} dx$ são simples de calcular enquanto que integrais como $\int \frac{ax^3+bx^2+cx+d}{dx^2+ex+f} dx$ não são. Na verdade, a integral de uma função dada pela razão entre dois polinômios com grau maior ou igual a 2 é bastante complicada sem essa técnica.

O que faremos nesse capítulo é uma revisão dos aspectos algébricos necessários para o uso dessa técnica, sem nos preocuparmos de fato com o cálculo das integrais. O seu professor de MA111 cobrirá esse último aspecto.

\section{Quando usar}

Este capítulo deve ser resolvido antes do professor de MA111 abordar integração por frações parciais.

\section{Conteúdos anteriores}

Todo o conteúdo deste capítulo gira em torno de manipulações algébricas com frações envolvendo polinômios. Portanto, é fundamental que você tenha estudado o capítulo sobre polinômios deste material. Especificamente, os tópicos a seguir são pré-requisitos para este capítulo:
\begin{itemize}
 \item Igualdade de polinômios;
 \item Divisão de polinômios.
\end{itemize}

Esses tópicos não serão cobertos durante as atividades de tutoria. Se você acha que não sabe o suficiente sobre algum deles, sugerimos que volte ao capítulo sobre polinômios deste material e os revise.

\newpage

\section{Questões}

O nosso objetivo ao longo das próximas questões será transformar a expressão $\frac{x^3+5x}{x^2+3x+2}$ em uma expressão que seja composta por uma soma de um polinômios e frações que tenham polinômios de primeiro grau como denominadores, ou seja, algo como $p(x)+\frac{a}{bx+c}$ (talvez seja necessáro mais de uma fração com formato semelhante).

\subsection*{Raízes}

Vamos começar fazendo algumas transformações básicas.

\begin{questao}
Considerando o numerador e o denominador da fração $\frac{x^3+5x}{x^2+3x+2}$ separadamente, responda:
\begin{enumerate}[a)]
\item Quais são as raízes de $x^2+3x+2$?
\item Escreva $x^2+3x+2$ na forma fatorada.
\item Quais são as raízes de $x^3+5x$?
\item Escreva $x^3+5x$ na forma fatorada.
\item Rescreva a fração com as formas fatoradas.
\end{enumerate}
\end{questao}

Note que não há nenhum fator em comum entre numerador e denominador para que possamos simplificar a fração dada.

\subsection*{Fração imprópria}

Você deve se lembrar do conceito de fração imprópria. Um exemplo é a fração $\frac{7}{3}$. Ela é chamada de imprópria porque 7 é maior do que 3, portanto, ela pode ser rescrita como um número inteiro mais uma fração própria:$\frac{7}{3}=\frac{6+1}{3}=\frac{6}{3}+\frac{1}{3}=2+\frac{1}{3}$.

O mesmo pode ser dito da nossa fração, como o numerador tem grau maior do que o denominador, se efetuarmos a divisão entre os polinômios vamos obter uma expressão equivalente mas com uma parte fracionária mais simples.

\begin{questao}
Efetue a divisão de ${x^3+5x}$ por ${x^2+3x+2}$. Caso você não se lembre exatamente como proceder, volte para a questão X do capítulo Y onde você pode ver um exemplo.
\end{questao}

Você deve ter obtido o quociente $(x-3)$ e resto $(12x+6)$. Isso significa que ${x^3+5x}$ é igual a $(x^2+3x+2)(x-3)+(12x+6)$. Se você está estranhando essa igualdade, leia a questão seguinte do capítulo Y. 

Dessa forma, podemos rescrever a fração inicial como mostrado abaixo:

\begin{align*}
\frac{x^3+5x}{x^2+3x+2} &= \frac{(x^2+3x+2)(x-3)+(12x+6)}{x^2+3x+2} \\
&=\frac{(x^2+3x+2)(x-3)}{x^2+3x+2}+\frac{12x+6}{x^2+3x+2} \\
&=(x-3)+\frac{12x+6}{x^2+3x+2} \\
\end{align*}

Veja que nessa expressão, a primeira parcela é simples de integrar e a segunda já tem um numerador com grau menor do que a original. Vamos focar em $\frac{12x+6}{x^2+3x+2}$ deste ponto em diante.

\subsection*{Mudando as frações}

Se você recuperar a forma fatorada obtida na primeira questão, podemos escrever $\frac{12x+6}{x^2+3x+2}$ na forma $\frac{12x+6}{(x+1)(x+2)}$. Essa forma sugere que essa fração poderia ser rescrita como uma soma de frações com $(x+1)$ e $(x+2)$ como denominadores, ou seja, como uma soma do tipo $\frac{A}{x+1}+\frac{B}{x+2}$ para algum valor de $A$ e $B$.

\begin{questao}
Para fins de prática, efetue a soma $\frac{3}{x+1}+\frac{4}{x+2}$.
\end{questao}

Voltando à nossa fração de interesse, $\frac{12x+6}{(x+1)(x+2)}$, como o numerador é de grau 1, é razoável esperar que existam números reais $A$ e $B$ que façam com que ela possa ser rescrita na forma $\frac{A}{x+1}+\frac{B}{x+2}$.

\begin{questao}
Efetue a soma $\frac{A}{x+1}+\frac{B}{x+2}$ e escreva o numerador na forma de uma binômio, ou seja, no formato $mx+n$.
\end{questao}

Você deve ter obtido a fração $\frac{(A+B)x+(2A+b)}{x^2+3x+2}$ na questão acima. Se igualarmos essa fração à $\frac{12x+6}{x^2+3x+2}$ (note que os denominadores sao iguais), concluímos que $(A+B)x+(2A+b)=12x+6$.

\begin{questao}
Determine o valor de $A$ e $B$ para a igualdade de polinômios $(A+B)x+(2A+b)=12x+6$ seja satisfeita.
\end{questao}

Isso significa que $\frac{12x+6}{x^2+3x+2}$ é igual a $\frac{-6}{x+1}+\frac{18}{x+2}$. Essas duas frações são chamadas de frações parciais.

Em conclusão, a fração que tínhamos inicialmente pode ser rescrita seguindo as transformações abaixo:

\begin{align*}
\frac{x^3+5x}{x^2+3x+2} = (x-3)+\frac{12x+6}{x^2+3x+2} = (x-3)+\frac{-6}{x+1}+\frac{18}{x+2}
\end{align*}

Você pode verificar se a igualdade acima está correta desenvolvendo a expressão mais à direita e verificando se você chega na forma dada inicialmente (sugiro que isos seja feito caso você não se sente confortável com as manipulações algérbica realizadas até aqui).

Finalmente, note que a fração inicial não era simples de integrar, mas a obtida ao final do processo é (basta fazer a troca $u=x+1$ para a primeira fração e $t=x+2$ para a segunda).

\subsection*{Três casos iniciais}

Para compreender o processo como um todo, vamos fazer três casos incompletos. No primeiro, é necessário apenas a primeira parte, ou seja, dividir os polinômios de modo a não termos mais uma fração imprópria. No segundo podemos ir direto para as frações parciais, sem precisar dividir os polinômios. O terceiro também será incompleto, mas cabe a você concluir onde.

\begin{questao}
Transforme a fração $\frac{x^3+6x^2+11x+6}{x^2-2}$ de modo que seja uma soma de um polinômio com uma fração algébrica não imprópria.
\end{questao}

\begin{questao}
Transforme a fração $\frac{2x+1}{x^2+3x-4}$ em uma soma de frações cujos denominadores sejam binômios do tipo $ax+b$ e e numeradores sejam números reais.
\end{questao}

\begin{questao}
Rescreva a fração $\frac{x^3+x}{x-1}$ no formato mais simples para integração que você puder.
\end{questao}

Note que na questão anterior não foi necessário obter as frações parciais, pois a fração restante depois da divisão de polinômios já possuía denominador do primeiro grau e um número real como numerador.

\subsection*{Um caso completo}

Agora vamos resolver um caso completo. Não se preocupe em terminar rapidamente. A intenção aqui é compreender cada etapa do processo para que, quando o tópico for abordado em MA111, você saiba se orientar dentre os diversos casos que serão discutidos.

\begin{questao}
Vamos transformar a fração $\frac{(x+5)(x+2)(x-2)(x-1)}{x^3+4x^2-x^2-4x}$ em uma soma que envolva frações com denominadores mais simples.
\begin{enumerate}[a)]
\item Fatore o denominador e rescreva a fração simplificando fatores se possível.
\item Desenvolva os fatores que sobraram no numerador e rescreva a fração.
\item Como o grau do numerador é maior do que o do denominador, faça a divisão do numerador pelo denominador de modo a rescrever a fração dada como uma soma de um polinômio com uma fração com numerador de grau menor.
\item Use o quociente e o resto obtido na questão acima para rescrever o numerador da fração inicial e simplificá-la, como feito após a questão 2. Qual é a parte fracionária da forma simplificada?
\item Como a parte fracionária obtida tem numerador do primeiro grau e denominador do segundo grau que pode ser fatorado, iguale a parte fracionária a uma soma de frações do tipo $\frac{A}{mx+n}+\frac{B}{ux+v}$ usando os fatores restantes no denominador, como feito na questão 4. Quais são os valores de $A$ e $B$?
\item Qual é a forma final da fração dada inicialmente?
\end{enumerate}
\end{questao}

\section{Mensagem final}

O processo que você utilizou para resolver a questão anterior é a justificativa por trás da técnica de integração conhecida como \textbf{frações parciais}. Os exemplos discutidos pertencem, na verdade, a um mesmo caso (denominador com todas as raízes reais e diferentes). Nas aulas de Cálculo você verá como o método precisa ser ajustado quando o denominador possui raízes iguais ou quando alguma de suas raízes não é real. Porém, a essência da técnica é o que fizemos aqui: dividir os polinômios de modo a reduzir o grau do polinômio no numerador e depois tentar encontrar frações parciais cuja soma seja igual à parte fracionária restante.

Nossa sugestão é que agora você leia a seção 7.4 do livro \sugestao{Calculus} até o final do terceiro exemplo e então comece a resolver os exercícios da lista oficial de MA111 sobre fraoes parciais.

\newpage

\section{Gabarito}

Confira as respostas para as questões e \textbf{não se esqueça de registrar o seu progresso}.

\noindent\textbf{Questão 1:} a) $f'(x)=e^x+xe^x$, b) $g'(x)=\cos(x)-x\sin(x)$, c) $h'(x)=\sin(x)+x\cos(x)$, d) $i'(x)=ln(x)+1$, e) $j'(x)=2x.ln(x)+x$.

\noindent\textbf{Questão 2:} a) $x\sin(x)=\cos(x)-g'(x)$, b) $\int x\sin(x) dx = \sin(x)-x\cos(x)+c$.

\noindent\textbf{Questão 3:} $\int x\cos(x) dx = \cos(x)+x\sin(x)+c$.

\noindent\textbf{Questão 4:} $\int ln(x) dx = x.ln(x)-x+c$.

\noindent\textbf{Questão 5:} a) $f(x)=x.ln(x)$, b) $\int x.ln(x) dx = \frac{x^2 ln(x)}{2}-\frac{x^2}{4}+c$.

\noindent\textbf{Questão 6:} $\int x^5.ln(x) dx = \frac{x^6 ln(x)}{6}-\frac{x^6}{36}+c$.

\section{Registro de progresso}

Essa parte por enquanto fica com conteúdo vazio até que seja decidido como será feito o controle do progresso.

\vspace{5cm}

\end{document}
