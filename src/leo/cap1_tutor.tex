\documentclass[main_estudante.tex]{subfiles}

\begin{document}

\chapter{Matrizes 2x2}

\section{Questões diagnósticas}

\begin{diagnostico}
Seja $A=\begin{pmatrix}2 & 5 \\ 1 & 6\end{pmatrix}$ e $B=\begin{pmatrix}-2 & 4 \\ 3 & 5\end{pmatrix}$, calcule:
\begin{enumerate}[a)]
  \item $3A-B$
  \item $A \times B$ 
  \item $A^{-1}$
  \item $det(A)$.
\end{enumerate}
\end{diagnostico}

\section{Gabarito}

\textbf{Questão 1:} a) $\begin{pmatrix}8 & 11 \\ 0 & 12\end{pmatrix}$, b) $\begin{pmatrix}11 & 33 \\ 16 & 34\end{pmatrix}$, c) $\begin{pmatrix}6/7 & -5/7 \\ -1/7 & -2/7\end{pmatrix}$, d) $7$.

Acertar o terceiro item permite um salto substancial nas questões desse capítulo, por isso os valores números envolvidos não são simples. Portanto, seja rigoroso até mesmo com aspectos aritméticos da resolução neste item. Entretanto, erros dessa natureza nos demais itens podem ser relevados se lhe parecer coerente com as demais resoluções.

\section{Quadro de orientação}

\begin{center}
 \begin{tabular}{|c c c c |c|} 
 \hline
 1A & 1B & 1C & 1D & Onde começar\\
 \hline
 C & E & E & E & Questão 2 \\ 
 \hline
 C & C & E & E & Questão 3 \\ 
 \hline
 C & C & C & E & Questão 7 \\ 
 \hline
 C & C & C & C & Questão 8 \\ 
 \hline
\end{tabular}
\end{center}

\section{Comentários iniciais}

Nesta seção, serão trabalhados conteúdos de matrizes utilizando apenas matrizes $2x2$. Essa escolha visa reduzir o volume de manipulações algébricas e cálculos aritméticos e focar nos conceitos e procedimentos centrais. São eles:

\begin{itemize}
 \item Operações: soma, multiplicação por escalar e multiplicação de matrizes;
 \item Igualdade de matrizes;
 \item Matriz inversa: obtenção e significado;
 \item Cálculo de determinante e propriedades relacionadas a matriz inversa e sistemas lineares;
 \item Sistemas lineares $2x2$: resolução;
 \item Sistemas lineares $2x2$: notação matricial;
\end{itemize}

É muito importante tentar separar dificuldades relacionadas a manipulações algébricas e cálculos aritméticos de dificuldades conceituais ou de entendimento do procedimento em si (como o de multiplicação de matrizes). Fique atento a isso e tente deixar claro aos estudantes que qual tipo de dificuldade eles estão enfrentando.

\section{Questões comentadas}

\begin{questao}
Considerando as matrizes $A=\begin{pmatrix} 2 && 1 \\ -3 && 0 \end{pmatrix}$, $B=\begin{pmatrix} 3 && 5 \\ -2 && -1 \end{pmatrix}$ e $C=\begin{pmatrix} 5 && -1 \\ 1 && -2 \end{pmatrix}$, realize as operações indicadas abaixo.
\begin{enumerate}[a)]
\item $A+B$
\item $C-B$
\item $5C+A$
\item $3C+\begin{pmatrix} 5 && 0 \\ \frac{2}{3} && -1 \end{pmatrix}$
\end{enumerate}
\end{questao}

Note que o item d traz uma fração como elemento de uma das matrizes. Embora esse tópico nao seja coberto no material da tutoria, talvez alguns estudantes precisem de ajudar neste ponto. A soma pode ser resolvida igualando-se os denominadores: $-6+\frac{2}{3}=-\frac{6}{1}+\frac{2}{3}=-\frac{18}{3}+\frac{2}{3}=-\frac{16}{3}$ .

\begin{questao}
Considerando as matrizes $D=\begin{pmatrix} 2 && 3 \\ 4 && 5 \end{pmatrix}$, $E=\begin{pmatrix} 7 && 8 \\ 9 && 10 \end{pmatrix}$ e $F=\begin{pmatrix} -2 && 1 \\ 0 && -3 \end{pmatrix}$, realize as operações indicadas abaixo.
\begin{enumerate}[a)]
\item $D \times E$
\item $E \times F$
\item $F \times E$
\item $F \times D$
\end{enumerate}
\end{questao}

Frações foram evitadas nessa questão para deixar o foco no procedimento de multiplicação, que é mais intrincado que o da soma e multiplicação por escalar. Os itens b e c exploram a questão da não comutatividade da multiplicação entre matrizes. O texto logo após a questão trata desse aspecto.

Note que o item d pede $F \times D$ e não $D \times F$ (ordem diferente da que as matrizes são apresentadas no enunciado).

\begin{questao}
Usando as matrizes definidas nas duas questões anteriores, realize os cálculos indicados abaixo.
\begin{enumerate}[a)]
\item $A^2$, lembre-se de que $A^2=A \times A$.
\item $B \times C - D^2$
\item $F \times (E-A)$ 
\end{enumerate}
\end{questao}

Nesta questão a ordem em que as operações devem ser executadas é fundamental, não apenas a ordem das matrizes ao serem multiplicadas. No item c a propriedade distributiva pode ser utilizada, ou seja $F \times (E-A) = F \times E- F \times A$. Note que a matriz $F$ deve multiplicar $E$ e $A$ pela esquerda (pois multiplicava o parênteses pela esquerda). Entretanto, fazer a sutração primeiro é o caminho mais simples por diminuir o número de multiplicação que precisam ser realizadas.

\begin{questao}
Determine o valor das incógnitas em cada uma das equações abaixo.
\begin{enumerate}[a)]
\item $\begin{pmatrix} 2 && x \\ 3y-1 && 0 \end{pmatrix} = \begin{pmatrix} 2 && 3 \\ 7 && z^2-25 \end{pmatrix}$
\item $ 3 \times \begin{pmatrix} -1 && t \\ 0 && 3 \end{pmatrix} + \begin{pmatrix} 4 && 2 \\ \frac{1}{2} && 0,6 \end{pmatrix} = \begin{pmatrix} 1 && 8 \\ \frac{1}{2} && 9,6 \end{pmatrix} $
\end{enumerate}
\end{questao}

Nesse momento, as igualdades envolvendo apenas elementos das matrizes e não as matrizes como um todo. No item a, temos 3 equações (note que o primeiro elemento das duas matrizes é igual) em crescente ordem de dificuldade do ponto de vista das equações envolvidas. No item b, há uma única equação, mas para resolvê-la é necessário operar um pouco com as matrizes. Pode valer a pena mencionar com os estudantes a possibilidade de fcar exclusivamente no elemento (1,2) da matriz, já que apenas ele contem uma incógnita.

\begin{questao}
Determinem as matrizes $M$ e $N$ que satisfazem as igualdades a seguir.
\begin{enumerate}[a)]
\item $ \begin{pmatrix} 1 && 0 \\ 2 && 1 \end{pmatrix} \times M = \begin{pmatrix} 2 && -3 \\ \frac{1}{2} && 4 \end{pmatrix} $
\item $ \begin{pmatrix} 1 && 2 \\ 0 && 2 \end{pmatrix} \times N = \begin{pmatrix} 1 && 0 \\ 0 && 1 \end{pmatrix} $ 
\end{enumerate}
\end{questao}

Agora as igualdades envolvem matrizes inteiras, de modo que todos os seus elementos devem ser obtidos. Não se preocupe com a notação $m_{i,j}$ nesse momento, ela será introduzida mais adiante. Diferentemente da questão anterior, as equações precisam ser resolvidas em sistemas (ainda que bastante simples) e não isoladamente.

\begin{reflita} Antes de resolver os itens abaixo, descreva textualmente como você vai proceder para resolvê-los. Tente cobrir todos os passos do processo, do início até a obtenção da solução.
\end{reflita}

Essa é a primeira vez que uma pergunta do tipo Reflita é proposta aos estudantes. A intenção é praticar  meta-cognição, ou seja, a capacidade de conscientemente planejar, monitorar e avaliar as ações para a resolução de uma questão. Não é necessário cobrar um texto perfeito, descrições esquemáticas devem ser aceitas e até mesmo incentivadas. Porém, é importante que a descrição seja o mais completa possível em termos das etapas para resolução da questão. Esperamos que essas questões tragam à tona aspectos matemáticos que até então estavam ocultos.

\begin{questao}
Obtenha a matriz inversa da:
\begin{enumerate}[a)]
\item matriz $A$ da questão 1.
\item matriz $D$ da questão 2.
\item $ \begin{pmatrix} 2 && -1 \\ -6 && 3 \end{pmatrix}$.
\end{enumerate}
\end{questao}

Os cálculos nesta questão vão começar a ficar longos e envolver frações. Portanto, é esperado que os estudantes cometam erros procedimentais. Obviamente é importante corrigir estes erros, mas certifique-se de que os estudantes entenderam os aspectos conceituais por trás dessa questão.

\begin{itemize}
 \item Como obter a matriz inversa $M^{-1}$ de uma matriz dada $M$: resolvendo as equações resultantes da igualdade $M \times M^{-1} = I$;
 \item Essa igualdade normalmente leva a sistemas de equações e não equações isoladas.
\end{itemize}

\begin{questao}
Calcule o determinante das seguinte matrizes.
\begin{enumerate}[a)]
\item matriz $A$ da questão 1.
\item matriz $E$ da questão 2.
\item $\begin{pmatrix} \frac{1}{3} && 1 \\ \frac{1}{2} && 2 \end{pmatrix}$
\item matriz do item c da questão anterior.
\end{enumerate}
\end{questao}

Nenhuma surpresa nessa questão. Não há necessidade de abordar determinante de matrizes maiores nesse momento (talvez mencionar que este procedimento é um caso particular para matrizes $2x2$). 

Entretanto, o texto logo após a questão trata um pouco sobre as diferenças entre uma descrição informal de um resultado matemático e um teorema. Certifique-se de que os estudantes leram o texto, mas a intenção agora não é promover grandes discussões, apenas criar familiaridade.

\begin{questao}
Considere a matriz $M=\begin{pmatrix}1 & 2 \\ 3 & m\end{pmatrix}$.
\begin{enumerate}[a)]
\item Qual deve ser o valor de $m$ para que o determinante dessa matriz seja igual a 0?
\item Tente obter a matriz $M^{-1}$ para o valor de $m$ obtido no item anterior.
\item Escolha um valor para $m$ que seja diferente do obtido no item a e obtenha $M^{-1}$ para esse valor.
\end{enumerate}
\end{questao}

Mais uma vez, não há nenuma surpresa nessa questão. Incentive os estudantes a comparar os resultados obtidos no item c tendo a seguinte conclusão em mente: o determinante é igual a 0 apenas se $m=6$.

\begin{questao}
Resolva os sistemas lineares dados a seguir.
\begin{enumerate}[a)]
\item $\begin{pmatrix}2 & 5 \\ 4 & -3\end{pmatrix} \times \begin{pmatrix}x \\ y\end{pmatrix} = \begin{pmatrix}12 \\ -2\end{pmatrix}$
\item $\begin{pmatrix}1 & -1 \\ 0 & 4\end{pmatrix} \times \begin{pmatrix}x \\ y\end{pmatrix} = \begin{pmatrix}1 \\ 2\end{pmatrix}$
\end{enumerate}
\end{questao}

Trata-se apenas de uma questão para prática e familiarização com a representação de sistemas lineares na forma de matrizes.

\begin{questao}
Considere a matriz $S=\begin{pmatrix}2 & 4 \\ -4 & -8\end{pmatrix}$
\begin{enumerate}[a)]
\item Calcule $Det(S)$.
\item Resolva o sistema $S \times \begin{pmatrix}x \\ y\end{pmatrix} = \begin{pmatrix}6 \\ -15\end{pmatrix}$
\end{enumerate}
\end{questao}

Os estudantes podem estranhar o resultado do item b (sistema sem solução). Nesse caso, você pode mostrar a eles que se a primeira equação for multiplicada por $-2$ a equação resultante é contraditória com a segunda equação.

\begin{questao}
Use o critério discutido acima para determinar se os sistemas abaixo possuem ou não solução única.
\begin{enumerate}[a)]
\item $\begin{pmatrix}-1 & 3 \\ 4 & 2\end{pmatrix} \begin{pmatrix}x \\ y\end{pmatrix} = \begin{pmatrix}7 \\ 5\end{pmatrix}$
\item $\begin{pmatrix} 9 & -6 \\ 6 & -4\end{pmatrix} \begin{pmatrix}x \\ y\end{pmatrix} = \begin{pmatrix}0 \\ 1\end{pmatrix}$
\end{enumerate}
\end{questao}

Trata-se apenas de uma questão para prática do cálculo de determinantes e fixação do resultado envolvendo determinantes e sistemas lineares.

Um detalhe não foi discutido em profundidade no material do estudante: determinante igual a zero pode resultar em um sistema impossível ou um sistema possível e indeterminado (admite infinitas soluções). Um exemplo desse segundo caso pode ser obtido mudando o elemento $-15$ da matriz do item 5 questão 10 para $-12$. A distinção entre esses dois casos não pode ser feita via determinante.

\begin{questao}
Considere a matriz $M=\begin{pmatrix}2 & -1 \\ 1 & 3\end{pmatrix}$.
\begin{enumerate}[a)]
\item Calcule $Det(M)$.
\item Obtenha $M^{-1}$
\item Resolva $M \times \begin{pmatrix}x \\ y\end{pmatrix} = \begin{pmatrix}6 \\ 17\end{pmatrix}$
\end{enumerate}
\end{questao}

Essa questão abarca os tópicos mais importantes deste capítulo do ponto de vista do suporte para Geometria Analítica.

\section{Gabarito}

\noindent\textbf{Questão 1:} a) $\begin{pmatrix} 5 & 6 \\ -5 & 1\end{pmatrix}$, b) $\begin{pmatrix} 2 & -6 \\ 3 & -1\end{pmatrix}$, c) $\begin{pmatrix} 27 & -4 \\ 2 & -10\end{pmatrix}$, d) $\begin{pmatrix} 14 & 15 \\ -16/3 & -4\end{pmatrix}$.

\noindent\textbf{Questão 2:} a) $\begin{pmatrix} 41 & 46 \\ 73 & 82\end{pmatrix}$, b) $\begin{pmatrix} -14 & -17 \\ -18 & -21\end{pmatrix}$, c) $\begin{pmatrix} -5 & -6 \\ -27 & -30\end{pmatrix}$, d) $\begin{pmatrix} 0 & -1 \\ -12 & -15\end{pmatrix}$.

\noindent\textbf{Questão 3:} a) $\begin{pmatrix} 1 & 2 \\ -6 & -3\end{pmatrix}$, b) $\begin{pmatrix} 4 & -34 \\ -39 & -33\end{pmatrix}$, c) $\begin{pmatrix} 2 & -4 \\ -36 & -30\end{pmatrix}$.

\noindent\textbf{Questão 4:} a) $x=3$ e $y=8/3$ e $z= \pm 5$, b) $t=2$.

\noindent\textbf{Questão 5:} a) $\begin{pmatrix} 2 & -3 \\ -7/2 & 10\end{pmatrix}$, b) $\begin{pmatrix} 1 & -1 \\ 0 & 1/2\end{pmatrix}$.

\noindent\textbf{Questão 6:} a) $\begin{pmatrix} 3/2 & -1/2 \\ -2 & 1\end{pmatrix}$, b) $\begin{pmatrix} -5/2 & 3/2 \\ 2 & -1\end{pmatrix}$, c) o sistema não tem solução.

\noindent\textbf{Questão 7:} a) $3$, b) $-2$, c) $\frac{1}{6}$, d) $0$.

\noindent\textbf{Questão 8:} a) $m=6$, c) se $m=1$ então a matriz inversa será $\begin{pmatrix} -1/5 & 2/5 \\ 3/5 & -1/5 \end{pmatrix}$.

\noindent\textbf{Questão 9:} a) $x=1$ e $y=2$, b) $x=\frac{3}{2}$ e $y=\frac{1}{2}$.

\noindent\textbf{Questão 10:} a) $0$, b) sistema sem solução.

\noindent\textbf{Questão 11:} a) sim, b) não.

\noindent\textbf{Questão 12:} a) $7$, b) $\begin{pmatrix} 3/7 & 1/7 \\ -1/7 & 2/7 \end{pmatrix}$, c) $x=5$, $y=4$.

\section{Rumo ao livro-texto}

O objetivo é discutir uma questão do que tipo ``mostre que'', tão comuns no ensino superior mas raramente abordadas no ensino médio. O ponto central a ser enfatizado aqui é a necessidade de não usar a hipótese (aquilo que se quer mostrar verdadeiro) ao longo da demonstração. No caso da questão resolvida e da proposta, utilizar a hipótese não facilita a resolução, mas incorre em um inconsistência lógica. Portanto, basta cuidado na apresentação da resolução para que isso não aconteça. Ao checar a resolução da questão proposta, seja bastante rigoroso com esse aspecto.

\section{Questões adicionais}

\begin{adicional}
Seja $A$ uma matriz diagonal de dimensões $2x2$
\begin{enumerate}[a)]
\item Calcule $A^2$ e $A^3$.
\item O que deve ocorrer com a matriz $A^n$?
\end{enumerate}
\end{adicional}

Insista que os estudantes devem de fato escrever a resposta do item b tentando ser o mais claros e completos possível. Uma possiblidade é dizer que os elementos $b$ da matriz $B=A^n$ são dados por $b_{i,j}=a_{i,j}^n$, ou seja, a potência da matriz pode ser ``distribuída'' internamente para os elementos.

\begin{adicional}
Considere a matriz $M=\begin{pmatrix}2 & \sqrt{6} \\ \sqrt{2} & x\end{pmatrix}$.
\begin{enumerate}[a)]
\item Obtenha $x$ para que o determinante dessa matriz seja igual a $0$.
\item Obtenha $x$ para que o determinante dessa matriz seja igual a $\sqrt{3}$.
\item Obtenha a $M^{-1}$ para o valor de $m$ calculado no tem b.
\end{enumerate}
\end{adicional}

Essa questão não envolve conceitos novos, mas demanda manipulações algébricas com raízes quadradas. que podem gerar certas dificuldades.

\begin{adicional}
Considere o sistema $\begin{pmatrix}12 & a \\ 9 & 6\end{pmatrix} \times \begin{pmatrix}x \\ y\end{pmatrix} = \begin{pmatrix}4 \\ t\end{pmatrix}$.
\begin{enumerate}[a)]
\item Determine o valor de $a$ para que o sistema não admita solução única.
\item Considerando o valor de $a$ obtido no item anterior, qual deve ser o valor de $t$ para que o sistema tenha infinitas soluções?
\end{enumerate}
\end{adicional}

Essa questão envolve uma situação nova: a determinação do valor de $t$ para que o sistema tenha infinitas soluções. Esse caso ocorre quando as linhas do sistema são múltiplas uma da outra.

Caso os alunos tenham resolvido essas questões, sugerimos os exercícios das seções 1.6 (matrizes e operações), 2.6 (sistemas lineares) e 3.10 (determinantes e matrizes inversas) do livro \sugestao{Álgebra Linear}. Entretanto, essa ista contem questões com matrizes de dimensões maiores.

\end{document}
